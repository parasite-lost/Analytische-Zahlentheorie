\section*{Aufgabe 1}
\begin{enumerate}[(1)]
\item
$\gcd(d,a) = 1$ bedeutet, dass in $a$ keiner der Primteiler von $d$ vorkommt. Mit $d \mid ab$ folgt aber, dass alle Primteiler von $d$ mit ihrer Vielfachheit in $ab$ vorkommen. Also müssen diese in $b$ enthalten sein, also $d \mid b$.

\item
Für ein $d \in \mathbb{N}$ und alle $a, b \in \mathbb{N}$ gelte $d \nmid a, d \mid ab \Rightarrow d \mid b$.

Sei $d=1$:

Es gilt $1 \mid a$ für alle $a \in \mathbb{N}$. Die Voraussetzungen der Implikation sind also nicht erfüllt. Hier ist also ein Fehler in der Aufgabenstellung.

Sei $d$ zusammengesetzt:

Dann gibt es $a$ mit $d \nmid a$, wobei jedoch $a$ und $d$ gemeinsame Primteiler haben, also $d = p \cdot d'$ und $a = p \cdot a'$, hierbei sei $p$ das Produkt aller gemeinsamen Primteiler.
Mit $ d \mid ab$ gilt damit $b = d' \cdot b'$; es gibt ein $b$ für das $b'$ und $d$ keine gemeinsamen Primteiler hat, damit gilt $d \mid ab = p a' d' b' = d a' b'$, jedoch nicht $d \mid b$. Also gilt die Implikation für zusammengesetzte $d$ nicht für alle $a,b$.

Sei $d$ prim:

Dann gilt $\gcd(d,a) = 1$ und die Situation ist wie in (1). 

Damit gilt die Behauptung (teilweise).
\end{enumerate}


\section*{Aufgabe 2}
\[ \frac{3^{4n +2}-1}{8} = \frac{(3^{2n+1} - 1)(3^{2n+1}+1)}{8} = \frac{(3 \cdot 9^n - 1)(3 \cdot 9^n+1)}{8}\]
Da $9 \equiv 1 \bmod 4$ gilt, ist auch $9^n \equiv 1 \bmod 4$. Somit ist 
\[ 3 \cdot 9^n - 1 \equiv 3 - 1 \equiv 2 \mod 4 \]
\[ 3 \cdot 9^n + 1 \equiv 3 + 1 \equiv 0 \mod 4 \]
Damit ist 
\[  \frac{(3 \cdot 9^n - 1)(3 \cdot 9^n+1)}{8} = \frac{3 \cdot 9^n - 1}{2} \cdot \frac{3 \cdot 9^n + 1}{4} \]
ein Produkt zweier natürlicher Zahlen.

Damit bleibt noch zu zeigen, dass beide Faktoren ungerade sind.
Angenommen der erste Faktor sei gerade: 
\[  \frac{3 \cdot 9^n - 1}{2} \equiv 0 \mod 2 \]
Dann gilt:
\[  \frac{3 \cdot 9^n}{2} = 2\cdot x + \frac{1}{2} \]
\[ 3 \cdot 9^n = 4\cdot  x + 1 \]
mit einem $x \in \mathbb{N} $. Da aber $3 \cdot 9^n \equiv 3 \not\equiv 1 \equiv 4\cdot x + 1 \bmod 4$ gilt,
führt dies zu einem Widerspruch.
Angenommen der zweite Faktor sei gerade:
\[ \frac{3 \cdot 9^n + 1}{4} \equiv 0 \mod 2\]
Dann gilt:
\[  \frac{3 \cdot 9^n}{4} = 2\cdot x + \frac{7}{4} \]
\[ 3 \cdot 9^n = 8x + 7 \]
mit einem $x \in \mathbb{N}$. Da aben $3 \cdot 9^n \equiv 3 \not\equiv 7 \equiv 8\cdot x + 7 \bmod 8$ gilt,
führt dies zu einem Widerspruch.

Somit sind beide Faktoren der Zahl ungerade und damit die Zahl selbst ungerade.
Also ist $ \frac{3^{4n +2}-1}{8}$ eine ungerade zusammengesetzte natürliche Zahl.


\section*{Aufgabe 3}
\begin{enumerate}[(1)]
\item 
\lstset{language=Mathematica}
\begin{lstlisting}
(* Liste der kleinsten Primteiler von n! - 1, 3 <= n <= 20 *)
Table[Divisors[n! - 1][[2]], {n, 3, 20}]
  {5, 23, 7, 719, 5039, 23, 11, 29, 13, 479001599, 1733, 87178291199,
  17, 3041, 19, 59, 653, 124769}
\end{lstlisting}
\item Der Satz von Wilson lautet $ p \text{ prim} \Leftrightarrow p \mid (p-1)! + 1 $.
Insbesondere teilt offensichtlich auch keine Zahl kleiner als $p$ die Zahl $  (p-1)! + 1 $. Desweiteren
teilt für $p \geq 5$ $p$ auch $ (p-1)! -  p + 1$.
\[ p \mid (p-1)! - p + 1 = ((p-2)! - 1)(p-1) \]
$p \nmid p-1$, also muss $p \mid (p-2)! - 1 $ gelten. Da $(p-2)!-1$ von keiner Zahl $\leq p-2$ geteilt wird,
$p-1$ keine Primzahl ist, ist auch $p$ weiterhin wie gewünscht der kleinste Primteiler.
\end{enumerate}

\section*{Aufgabe 4}
\begin{enumerate}[(1)]
\item
$p_1\cdots p_{n-1} - 1 \equiv -1 \mod p_i$ für alle $ 1 \leq i \leq n-1$; diese Zahl ist also nicht durch eine der $n-1$ ersten Primzahlen teilbar.
Da jede natürliche Zahl eine eindeutige Darstellung durch ihre Primteiler hat, muss es eine Primzahl $p_n \leq p_1\cdots p_{n-1} - 1$ geben, die in der Darstellung dieser Zahl als Primzahl vorkommt.
Damit gilt die Behauptung.

\item
I.V.: $p_n \leq 2^{2^{n-1}}$ für alle $n \geq 1$.

I.A.: $n = 1$, $p_n = p_1 = 2 \leq 2^{2^{n-1}} = 2^{2^0} = 2$

I.S.: Gelte I.V. für ein $n \in \mathbb{N}$, dann folgt mit (1):
\[ p_n \leq p_1 \cdots p_{n-1} - 1 = \prod_{k=1}^{n-1}{ 2^{2^{k-1}}} - 1 =  2^{\sum_{k=0}^{n-2}2^k} - 1 = 2^{\frac{1- 2^{n-1}}{1-2}} - 1 = 2^{2^{n-1}} - 1 \leq 2^{2^{n-1}}\]

\item
Mit (2) hat man für $ x= 2^{2^{n-1}} \geq p_n$ einen funktionalen Zusammenhang zwischen $x$ und $n$, sodass es mindestens n Primzahlen $\leq$ x gibt.

Damit kann man folgende Funktion konstruieren:
\[ \frac{\log 2^{2^{n-1}}}{\log 2} = 2^{n-1} \]
\[ \frac{\log \frac{\log 2^{2^{n-1}}}{\log 2}}{\log 2} = \frac{\log\log 2^{2^{n-1}} - \log\log 2}{\log 2} = n-1 \]
Mit $\log\log 2 < \log 2$ erhält man damit:
\[ n+1 \geq \frac{\log\log{2^{2^{n-1}}}}{\log 2} \geq n \]
Damit folgt sofort die Behauptung.


\end{enumerate}


\section*{Aufgabe 5}

Für $n \geq 12$ sind auf jeden Fall die Zerlegungen $n = a + b = 4 + (n-4) = 6 + (n-6) = 8 + (n-8)$ möglich; $4, 6, 8$ sind zusammengesetzte Zahlen, $n-4, n-6, n-8$ könnten jedoch Primzahlen sein; da es jedoch keine weiteren Primzahldrillinge als $3, 5, 7$ gibt, erfüllt eine dieser Zerlegungen auf jeden Fall die gewünschten Bedingungen, wenn $n-8 > 3$ gilt, was mit $n \geq 12$ immer der Fall ist.