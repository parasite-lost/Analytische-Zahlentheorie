\section*{Aufgabe 1}

\section*{Aufgabe 2}

\section*{Aufgabe 3}
\begin{enumerate}[(1)]
\item 
\lstset{language=Mathematica}
\begin{lstlisting}
(* Liste der kleinsten Primteiler von n! - 1, 3 <= n <= 20 *)
Table[Divisors[n! - 1][[2]], {n, 3, 20}]
  {5, 23, 7, 719, 5039, 23, 11, 29, 13, 479001599, 1733, 87178291199,
  17, 3041, 19, 59, 653, 124769}
\end{lstlisting}
\item Der Satz von Wilson lautet $ p \text{ prim} \Leftrightarrow p \mid (p-1)! + 1 $.
Insbesondere teilt offensichtlich auch keine Zahl kleiner als $p$ die Zahl $  (p-1)! + 1 $. Desweiteren
teilt für $p \geq 5$ $p$ auch $ (p-1)! -  p + 1$.
\[ p \mid (p-1)! - p + 1 = ((p-2)! - 1)(p-1) \]
$p \nmid p-1$, also muss $p \mid (p-2)! - 1 $ gelten. Da $(p-2)!-1$ von keiner Zahl $\leq p-2$ geteilt wird,
$p-1$ keine Primzahl ist, ist auch $p$ weiterhin wie gewünscht der kleinste Primteiler.
\end{enumerate}

\section*{Aufgabe 4}

\section*{Aufgabe 5}