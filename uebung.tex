\documentclass[a4paper]{article}
\pagestyle{empty}
\usepackage[utf8]{inputenc}
\usepackage[ngerman]{babel}
\usepackage{a4wide}
\usepackage{enumerate}
\usepackage{listings}
\usepackage{color}
\usepackage{fancyhdr}
\usepackage{subfig}
\usepackage{tikz}
\usepackage[colorlinks=true,linkcolor=black,a4paper=true]{hyperref}
\usepackage{amsmath, amsthm, amssymb}

\title{Analytische Zahlentheorie\\
       Sommersemester 2011 \\
       \small Lösungen zu den Übungsaufgaben}
\author{Ulrich Dorsch\\ FAU Erlangen-Nürnberg}
\lhead{Übung zur Vorlesung\\ Analytische Zahlentheorie}
\rhead{Ulrich Dorsch, Inf. M. Sc., 21325920}

\makeatletter
% Different format for headings
\def\section{\@startsection{section}{1}%
  \z@{1.7\baselineskip\@plus\baselineskip}{.5\baselineskip}%
  {\large\scshape\centering}}
% This does spacing around caption.
\setlength{\abovecaptionskip}{0.5em}
\setlength{\belowcaptionskip}{0.5em}
% This does justification (left) of caption.
\long\def\@makecaption#1#2{%
  \vskip\abovecaptionskip
  \sbox\@tempboxa{#2}%
  \ifdim \wd\@tempboxa >\hsize
    #2\par
  \else
    \global \@minipagefalse
    \hb@xt@\hsize{\box\@tempboxa\hfill}%
  \fi
  \vskip\belowcaptionskip}
\makeatother

\newcommand{\hk}{\mathcal{H}_{k}}
\newcommand{\hkp}{\mathcal{H}_{k+1}}
\newcommand{\im}{\textit{i}}
\newcommand{\mat}[2]{\begin{bmatrix}#1\\#2\end{bmatrix}}
\newcommand{\bra}[1]{\langle #1 |}
\newcommand{\ket}[1]{| #1 \rangle}
\newcommand{\bk}[1]{\langle #1 \rangle}
\newcommand{\Z}{\mathbb{Z}}

\begin{document}
\pagestyle{fancy}
%\maketitle
%\tableofcontents
%\section*{Aufgabe 1}

\section*{Aufgabe 2}

\section*{Aufgabe 3}
\begin{enumerate}[(1)]
\item 
\lstset{language=Mathematica}
\begin{lstlisting}
(* Liste der kleinsten Primteiler von n! - 1, 3 <= n <= 20 *)
Table[Divisors[n! - 1][[2]], {n, 3, 20}]
  {5, 23, 7, 719, 5039, 23, 11, 29, 13, 479001599, 1733, 87178291199,
  17, 3041, 19, 59, 653, 124769}
\end{lstlisting}
\item Der Satz von Wilson lautet $ p \text{ prim} \Leftrightarrow p \mid (p-1)! + 1 $.
Insbesondere teilt offensichtlich auch keine Zahl kleiner als $p$ die Zahl $  (p-1)! + 1 $. Desweiteren
teilt für $p \geq 5$ $p$ auch $ (p-1)! -  p + 1$  und $p$ ist weiterhin der kleinste Primteiler.
\[ p \mid (p-1)! - p + 1 = ((p-2)! - 1)(p-1) \]
$p \nmid p-1$, also muss $p \mid (p-2)! - 1 $ gelten. Auf Grund der obigen Betrachtung
ist $p$ auch wie gewünscht der kleinste Primteiler.
\end{enumerate}

\section*{Aufgabe 4}

\section*{Aufgabe 5}
\section*{Aufgabe 6}
\begin{enumerate}[(1)]
\item Auf Grund der Abgeschlossenheit bezüglich Multiplikation gilt für jedes Element $n \in S$, dass es als Produkt $n = d_1 d_2$, $d_1, d_2 \in S$, 
dargestellt werden kann und zwar entweder nur mit $(d_1, d_2) = (1, n)$, also $S$-Primzahl, oder mit $d_1 \neq 1 \neq d_2$. Für $d_1$ und $d_2$ gilt die selbe Aussage,
da diese Zahlen ebenfalls in $S$ sind. Ist also $n$ zusammengesetzt, erhält man zwei kleinere Zahlen $d_1, d_2 \in S$ für die die gleiche Argumentation gilt. So kann man
rekursiv immer kleinere Faktoren von $n$ konstruieren, da es nur endlich viele kleinere Zahlen als $n$ in $S$ gibt, sind die Faktoren irgendwann $S$-Primzahlen.
(Mir ist kein formal besserer Beweis eingefallen.)

\item Dies ist leicht zu konstruieren, indem man mittels Primzahlzerlegung in den natürlichen Zahlen und gleichzeitig in $S$ arbeitet:

\begin{eqnarray}  7 \cdot 7 \cdot 3 \cdot 11 & =& 7 \cdot 3 \cdot 7 \cdot 11 \\ 49 \cdot 33 &=& 21 \cdot 77  \end{eqnarray}

Die obere Zeile sind die natürlichen Primzahlen, die untere $S$-Primzahlen wie man leicht sehen kann, da deren natürlichen Primfaktoren nicht in $S$ sind.

\end{enumerate}


\section*{Aufgabe 7}
\begin{enumerate}[(1)]
\item

\item Mittels partieller Integration gilt:
\[ \int_2^x \frac{1}{\log t} dt = \int_2^x \frac{t'}{\log t} dt =  \left[ \frac{t}{\log t} \right]_2^x -  \int_2^x t \left(\frac{1}{\log t}\right)' dt = 
\frac{x}{\log x} - \frac{2}{\log 2} - \int_2^x t \left(  \frac{-1}{t (\log t)^2}  \right) dt = \]
\[ = \frac{x}{\log x} - \frac{2}{\log 2} + \int_2^x \frac{1}{(\log t)^2}  dt \]

\item Aus (1) und (2) folgt sofort:
\[ \int_2^x  \frac{1}{\log t} dt < \frac{x}{\log x} + \int_2^x \frac{1}{(\log t)^2}  dt =  \frac{x}{\log x} + \alpha(x) \frac{x}{(\log x)^2} \]
Mit $\underset{x \rightarrow \infty}{\lim} \alpha(x) = 1$, anders ausgedrückt: $\forall \varepsilon > 0 \exists x_0 \forall x > x_0 : |\alpha(x) - 1 | < \varepsilon$. Somit gilt auch
$\forall c > 1 \exists x_c \forall x > x_c : | \alpha(x) | < c (:= 1 + \varepsilon) $. Damit folgt sofort die Behauptung.

\end{enumerate}

\section*{Aufgabe 8}
\begin{enumerate}[(1)]

\item Für $n \geq 9$ ist $n-5 \geq 4$.Wenn man die Teilbarkeit durch 3 betrachtet, kommen also nur Vielfache der 3 vor:
Sei $n$ ungerade, dann sind $n, n-2, n-4$ Primzahlkandidaten, jedoch ist wie man leicht sieht mindestens eine durch 3 teilbar.
Sei $n$ gerade, dann sind $n-1, n-3, n-5$ Primzahlkandidaten, jedoch ist auch hier mindestens eine dieser Zahlen durch 3 teilbar.
Also können nur zwei Primzahlen in dem Intervall vorkommen.

\item Für $x \geq 9$ gilt also nach (1) $\pi(x) - \pi(x-6) \leq 2 = \frac{x}{3} - \frac{x-6}{3}$. Damit folgt sofort $\pi(x) \leq \frac{x}{3}$ für $x \geq 9$. 
Da $\pi(33) = 11$ ist, $\pi(34), \pi(35), \pi(36) = 11$, inspesondere $\pi(36) < \frac{36}{3}$ gilt für $x > 33$ die gewünschte Ungleichung $\pi(x) < \frac{x}{3}$,
 da es alle sechs aufeinanderfolgenden Zahlen maximal einen Zuwachs von zwei Primzahlen gibt.

\end{enumerate}


\section*{Aufgabe 9}
\begin{enumerate}[(1)]

\item Wie man leicht sieht ist $A(x)$ die Menge aller natürlichen Zahlen $\leq x$ dargestellt mit den Exponenten ihrer Primfaktorzerlegung, die höchste Potenz eines Primfaktors p kann
für eine beliebige Zahl maximal $\lfloor \log_p x \rfloor= \lfloor \frac{\log x}{\log p} \rfloor  \leq \lfloor \frac{\log x}{\log 2} \rfloor $ sein. Somit gilt in $A(x)$ für jedes $a_p \in \left\{0, ..., \lfloor \frac{\log x}{\log 2} \rfloor \right\} $.
Damit gilt offensichtlich $A(x) \subseteq B(x)$. Die Anzahl der Elemente von A(x) ist offensichtlich gleich der Anzahl der Zahlen $\leq x$, da die Primfaktorzerlegung eindeutig ist, also $|A(x)| = \lfloor x \rfloor$.
Bei $B(x)$ kann man leicht nachzählen, dass für jedes $p < x$, von denen es genau $\pi(x)$ gibt, für jedes $a_p$ jedes Element aus $\left\{0, ..., \lfloor \frac{\log x}{\log 2} \rfloor \right\}$ einmal ausgewählt wird, die kombinatorischen Möglichkeiten sind also
$| B(x) | = \left(1 + \lfloor \frac{\log x}{\log 2} \rfloor \right)^{\pi(x)} $.


\item Aus (1) folgt wegen  $A(x) \subseteq B(x)$:
\[ \lfloor x \rfloor \leq  \left(1 + \lfloor \frac{\log x}{\log 2} \rfloor \right)^{\pi(x)} \]
\[ \log \lfloor x \rfloor \leq \log \left(1 + \lfloor \frac{\log x}{\log 2} \rfloor \right)^{\pi(x)} = \pi(x) \log \left(1 + \lfloor \frac{\log x}{\log 2} \rfloor \right) \]
\[ \log (x-1) <  \pi(x) \log \left(  \frac{\log 2 + \log x}{\log 2}  \right)  = \pi(x) \log \left(  \frac{\log (2x)}{\log 2}  \right) \] 
\[ \pi(x) < \frac{\log (x-1)}{\log \log (2x) - \log \log 2} \]


\item
\end{enumerate}


\section*{Aufgabe 10}

\begin{enumerate}[(1)]
\item Es ist $2^n \equiv 1 \mod 3$ oder $\equiv 2 \mod 3$. Im ersten Fall ist $2^n - 1$ Vielfaches von 3, im zweiten $2^n + 1$. Somit kann man nur solche Primzahlzwillinge erhalten, wenn eine Zahl davon selbst die 3 ist. Dies ist nur für das Paar 3 und 5 möglich.

\item 
\lstset{language=Mathematica}
\begin{lstlisting}
Select[Table[{n, {3 2^n - 1, 3 2^n + 1}}, {n, 1, 20}],   
       PrimeQ[3 2^#[[1]] - 1] && PrimeQ[3 2^#[[1]] + 1] &]
  {{1, {5, 7}}, {2, {11, 13}}, {6, {191, 193}}, {18, {786431, 786433}}}
\end{lstlisting}
\end{enumerate}

\end{document}
