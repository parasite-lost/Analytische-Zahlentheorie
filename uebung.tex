\documentclass[a4paper]{article}
\pagestyle{empty}
\usepackage[utf8]{inputenc}
\usepackage[ngerman]{babel}
\usepackage{a4wide}
\usepackage{enumerate}
\usepackage{listings}
\usepackage{color}
\usepackage{fancyhdr}
\usepackage{subfig}
\usepackage{tikz}
\usepackage[colorlinks=true,linkcolor=black,a4paper=true]{hyperref}
\usepackage{amsmath, amsthm, amssymb}

\title{Analytische Zahlentheorie\\
       Sommersemester 2011 \\
       \small Lösungen zu den Übungsaufgaben}
\author{Felicitas Hetzelt\\Ulrich Dorsch\\ FAU Erlangen-Nürnberg}
\lhead{Übung zur Vorlesung\\ Analytische Zahlentheorie}
\rhead{
Felicitas Hetzelt, Inf. M. Sc., XXXXXXXX\\
Ulrich Dorsch, Inf. M. Sc., 21325920}

\makeatletter
% Different format for headings
\def\section{\@startsection{section}{1}%
  \z@{1.7\baselineskip\@plus\baselineskip}{.5\baselineskip}%
  {\large\scshape\centering}}
% This does spacing around caption.
\setlength{\abovecaptionskip}{0.5em}
\setlength{\belowcaptionskip}{0.5em}
% This does justification (left) of caption.
\long\def\@makecaption#1#2{%
  \vskip\abovecaptionskip
  \sbox\@tempboxa{#2}%
  \ifdim \wd\@tempboxa >\hsize
    #2\par
  \else
    \global \@minipagefalse
    \hb@xt@\hsize{\box\@tempboxa\hfill}%
  \fi
  \vskip\belowcaptionskip}
\makeatother

\newcommand{\hk}{\mathcal{H}_{k}}
\newcommand{\hkp}{\mathcal{H}_{k+1}}
\newcommand{\im}{\textit{i}}
\newcommand{\mat}[2]{\begin{bmatrix}#1\\#2\end{bmatrix}}
\newcommand{\bra}[1]{\langle #1 |}
\newcommand{\ket}[1]{| #1 \rangle}
\newcommand{\bk}[1]{\langle #1 \rangle}
\newcommand{\Z}{\mathbb{Z}}

\begin{document}
\pagestyle{fancy}
%\maketitle
%\tableofcontents
\section*{Aufgabe 1}

\section*{Aufgabe 2}

\section*{Aufgabe 3}
\begin{enumerate}[(1)]
\item 
\lstset{language=Mathematica}
\begin{lstlisting}
(* Liste der kleinsten Primteiler von n! - 1, 3 <= n <= 20 *)
Table[Divisors[n! - 1][[2]], {n, 3, 20}]
  {5, 23, 7, 719, 5039, 23, 11, 29, 13, 479001599, 1733, 87178291199,
  17, 3041, 19, 59, 653, 124769}
\end{lstlisting}
\item Der Satz von Wilson lautet $ p \text{ prim} \Leftrightarrow p \mid (p-1)! + 1 $.
Insbesondere teilt offensichtlich auch keine Zahl kleiner als $p$ die Zahl $  (p-1)! + 1 $. Desweiteren
teilt für $p \geq 5$ $p$ auch $ (p-1)! -  p + 1$  und $p$ ist weiterhin der kleinste Primteiler.
\[ p \mid (p-1)! - p + 1 = ((p-2)! - 1)(p-1) \]
$p \nmid p-1$, also muss $p \mid (p-2)! - 1 $ gelten. Auf Grund der obigen Betrachtung
ist $p$ auch wie gewünscht der kleinste Primteiler.
\end{enumerate}

\section*{Aufgabe 4}

\section*{Aufgabe 5}
\end{document}
