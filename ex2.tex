\section*{Aufgabe 6}
\begin{enumerate}[(1)]
\item Auf Grund der Abgeschlossenheit bezüglich Multiplikation gilt für jedes Element $n \in S$, dass es als Produkt $n = d_1 d_2$, $d_1, d_2 \in S$, 
dargestellt werden kann und zwar entweder nur mit $(d_1, d_2) = (1, n)$, also $S$-Primzahl, oder mit $d_1 \neq 1 \neq d_2$. Für $d_1$ und $d_2$ gilt die selbe Aussage,
da diese Zahlen ebenfalls in $S$ sind. Ist also $n$ zusammengesetzt, erhält man zwei kleinere Zahlen $d_1, d_2 \in S$ für die die gleiche Argumentation gilt. So kann man
rekursiv immer kleinere Faktoren von $n$ konstruieren, da es nur endlich viele kleinere Zahlen als $n$ in $S$ gibt, sind die Faktoren irgendwann $S$-Primzahlen.
(Mir ist kein formal besserer Beweis eingefallen.)

\item Dies ist leicht zu konstruieren, indem man mittels Primzahlzerlegung in den natürlichen Zahlen und gleichzeitig in $S$ arbeitet:

\begin{eqnarray}  7 \cdot 7 \cdot 3 \cdot 11 & =& 7 \cdot 3 \cdot 7 \cdot 11 \\ 49 \cdot 33 &=& 21 \cdot 77  \end{eqnarray}

Die obere Zeile sind die natürlichen Primzahlen, die untere $S$-Primzahlen wie man leicht sehen kann, da deren natürlichen Primfaktoren nicht in $S$ sind.

\end{enumerate}


\section*{Aufgabe 7}

\section*{Aufgabe 8}

\section*{Aufgabe 9}

\section*{Aufgabe 10}
